\documentclass[12pt,a4paper]{article}

\PassOptionsToPackage{hyphens}{url} % \sloppy\url{}
\usepackage[colorlinks=true,linkcolor=black,urlcolor=blue,citecolor=blue]{hyperref} % \href{https://www.youtube.com/}{Youtube} or \url{}
\parindent =0cm % No indentation
\usepackage[left=2cm,right=2cm,top=2.5cm,bottom=3cm]{geometry}

\title{Solving the time-independent Schrödinger equation for a hindered methyl rotor potential with \textit{QRotor.py}}
\author{Pablo Gila-Herranz}
\date{2024-03-27}

\begin{document}
\maketitle


\section*{Time-independent Schrödinger equation for a hindered methyl rotor potential}

The 1-dimensional time-independent Schrödinger equation is
$$
H\Psi(\varphi)=E\Psi(\varphi)
$$

The hamiltonian for a hindered methyl rotor can be expressed as a function of the angle $\varphi$.
It depends of the kinetic rotational energy and the potential energy:
$$
H = -B \frac{d^2}{d\varphi^2} + V(\varphi)
$$

with
$$
B = \frac{1}{2I}=\frac{1}{2\sum_{i}m_{i}r_{i}^{2}}
$$

The potential can be adjusted to the following form, where the coefficients are obtained via electronic calculation methods \cite{titov2023},
$$
V(\varphi)=c_{0}+c_{1}\sin(3\varphi)+c_{2}\cos(3\varphi)+c_{3}\sin(6\varphi)+c_{4}\cos(6\varphi)
$$


\section*{Finite Difference Method}

The time-independent Schrödinger equation is a second-order differential equation.
It can be solved numerically by discretizing it with the finite difference method \cite{finite_diff_python}.
This way, with a fine enough grid, the first derivative can be approximated as the slope,
$$
\frac{d\Psi}{d\varphi} = \frac{\Psi(\varphi+\Delta\varphi)-\Psi(\varphi)}{\Delta\varphi}
$$

Following the same procedure, the second derivative is
$$
\frac{d^2\Psi}{d\varphi^2} = \nabla^2\Psi = \frac{\frac{\Psi(\varphi+\Delta\varphi)-\Psi(\varphi)}{\Delta\varphi} - \frac{\Psi(\varphi)-\Psi(\varphi-\Delta\varphi)}{\Delta\varphi}}{\Delta\varphi} = \frac{\Psi(\varphi+\Delta\varphi)-2\Psi(\varphi)+\Psi(\varphi-\Delta\varphi)}{\Delta\varphi^2}
$$

This second derivative can be expressed in matrix form as
\[
    \nabla^2 = \frac{1}{\Delta\varphi^2}
    \left[ {\begin{array}{cccccc}
    -2      &  1     &  0     & \cdots &  0    &  0      \\
     1      & -2     &  1     &        &  0    &  0      \\
     0      &  1     & \ddots &        &       &  \vdots \\
     \vdots &        &        & \ddots &  1    &  0      \\
     0      &  0     &        &  1     & -2    &  1      \\
     0      &  0     & \cdots &  0     &  1    & -2      \\
    \end{array} } \right]
\]

The multiplication of this operator and the wavefunction vector yields the second derivative at every grid point.
Finally, imposing periodic boundary conditions, the first and last grid points are connected with an off-diagonal term,
\[
    \nabla^2 = \frac{1}{\Delta\varphi^2}
    \left[ {\begin{array}{cccccc}
    -2      &  1     &  0     & \cdots &  0    &  \bf{1} \\
     1      & -2     &  1     &        &  0    &  0      \\
     0      &  1     & \ddots &        &       &  \vdots \\
     \vdots &        &        & \ddots &  1    &  0      \\
     0      &  0     &        &  1     & -2    &  1      \\
     \bf{1} &  0     & \cdots &  0     &  1    & -2      \\
    \end{array} } \right]
\]

To build the potential energy operator, the energy at each grid point is set to equal the potential energy.
This results in a diagonal matrix, with the potential energy at each point along the diagonal:
\[
    V(\varphi) =
    \left[ {\begin{array}{cccc}
    V(\varphi_1) &  0            &  \cdots &  0          \\
     0           &  V(\varphi_2) &         &  \vdots     \\
     \vdots      &               &  \ddots &   \vdots     \\
     0           &   \cdots      &  \cdots & V(\varphi_N) \\
    \end{array} } \right]
\]

This way, the energy eigenvalues of the hindered methyl rotor can be obtained as the eigenvalues of the newly-constructed hamiltonian matrix. QRotor solves this eigenvalue problem with the SciPy package.

\bibliographystyle{IEEEtran} % Estilo
\bibliography{QRotor_refs} % documento .bib

\end{document}

