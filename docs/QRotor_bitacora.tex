\documentclass[12pt,a4paper]{article}

\PassOptionsToPackage{hyphens}{url} % \sloppy\url{}
\usepackage[colorlinks=true,linkcolor=black,urlcolor=blue,citecolor=blue]{hyperref} % \href{https://www.youtube.com/}{Youtube} or \url{}
\parindent =0cm % No indentation
\usepackage[left=2cm,right=2cm,top=2.5cm,bottom=3cm]{geometry}
\usepackage{amsmath}
\usepackage{float}
\usepackage{graphicx}
\usepackage{multicol}

% Quick pics
\newcommand{\NEWPIC}[3]{
\begin{figure}[H]
    \centering
    \includegraphics[width=#2\textwidth]{pics/#1}
    \caption{#3}
    \label{#1}
    \end{figure}
    }
% Without caption
\newcommand{\NEWPICC}[2]{
\begin{figure}[H]
    \centering
    \includegraphics[width=#2\textwidth]{pics/#1}
    \label{#1}
    \end{figure}
    }

\title{Solving the energies for a hindered methyl rotor potential with the \textit{QRotor} Python package}
\author{Pablo Gila-Herranz}
\date{\today}

\begin{document}
\maketitle
\tableofcontents
\thispagestyle{empty} % Quitar el formato en esta página
\newpage
\setcounter{page}{1} % Que no empiece la primera página en el índice


\section{Theoretical background}


\subsection{Hamiltonian for a hindered methyl rotor potential}


A hindered methyl rotor potential can be expressed as a function of the angle, $\varphi$. Thus its energies are given by the 1-dimensional time-independent Schrödinger equation,
$$
H\Psi(\varphi)=E\Psi(\varphi)
$$

The hamiltonian depends of the kinetic rotational energy and the potential energy,
\begin{equation} \label{hamiltonian}
H = -B \frac{d^2}{d\varphi^2} + V(\varphi)
\end{equation}

The kinetic rotational energy depends on the inertia $I=mr^2$ of each hydrogen, as
\begin{equation} \label{b}
B = \frac{\hbar}{2I}=\frac{\hbar}{2\sum_{i}m_{i}r_{i}^{2}}
\end{equation}

In QRotor the potential can be introduced as a custom array. It can also be adjusted to a function such as follows, where the coefficients were previously obtained via electronic calculation methods \cite{titov2023},
$$
V(\varphi)=c_{0}+c_{1}\sin(3\varphi)+c_{2}\cos(3\varphi)+c_{3}\sin(6\varphi)+c_{4}\cos(6\varphi)
$$


\subsection{Solving the time-independent Schrödinger equation with the Finite Difference Method}


The time-independent Schrödinger equation is a second-order differential equation.
It can be solved numerically by discretizing it with the finite difference method \cite{finite_diff_python}.
This way, with a fine enough grid, the first derivative can be approximated as the slope,
$$
\frac{d\Psi}{d\varphi} = \frac{\Psi(\varphi+\Delta\varphi)-\Psi(\varphi)}{\Delta\varphi}
$$

Following the same procedure, the second derivative is
$$
\frac{d^2\Psi}{d\varphi^2} = \nabla^2\Psi = \frac{\frac{\Psi(\varphi+\Delta\varphi)-\Psi(\varphi)}{\Delta\varphi} - \frac{\Psi(\varphi)-\Psi(\varphi-\Delta\varphi)}{\Delta\varphi}}{\Delta\varphi} = \frac{\Psi(\varphi+\Delta\varphi)-2\Psi(\varphi)+\Psi(\varphi-\Delta\varphi)}{\Delta\varphi^2}
$$

This second derivative can be expressed in matrix form as
\[
    \nabla^2 = \frac{1}{\Delta\varphi^2}
    \left[ {\begin{array}{cccccc}
    -2      &  1     &  0     & \cdots &  0    &  0      \\
     1      & -2     &  1     &        &  0    &  0      \\
     0      &  1     & \ddots &        &       &  \vdots \\
     \vdots &        &        & \ddots &  1    &  0      \\
     0      &  0     &        &  1     & -2    &  1      \\
     0      &  0     & \cdots &  0     &  1    & -2      \\
    \end{array} } \right]
\]

The multiplication of this operator and the wavefunction vector yields the second derivative at every grid point.
To impose periodic boundary conditions, the first and last grid points are connected with an off-diagonal term,
\[
    \nabla^2 = \frac{1}{\Delta\varphi^2}
    \left[ {\begin{array}{cccccc}
    -2      &  1     &  0     & \cdots &  0    &  \bf{1} \\
     1      & -2     &  1     &        &  0    &  0      \\
     0      &  1     & \ddots &        &       &  \vdots \\
     \vdots &        &        & \ddots &  1    &  0      \\
     0      &  0     &        &  1     & -2    &  1      \\
     \bf{1} &  0     & \cdots &  0     &  1    & -2      \\
    \end{array} } \right]
\]

Finally, to build the potential energy operator, the energy at each grid point is set to equal the potential energy.
This results in a diagonal matrix, with the potential energy at each point along the diagonal,
\[
    V(\varphi) =
    \left[ {\begin{array}{cccc}
    V(\varphi_1) &  0            &  \cdots &  0          \\
     0           &  V(\varphi_2) &         &  \vdots     \\
     \vdots      &               &  \ddots &   \vdots     \\
     0           &   \cdots      &  \cdots & V(\varphi_N) \\
    \end{array} } \right]
\]

This way, the energy eigenvalues of the hindered methyl rotor can be obtained as the eigenvalues of the newly-constructed hamiltonian matrix from equation \eqref{hamiltonian}. QRotor solves this eigenvalue problem with the shift-inverted mode of the ARPACK package, provided through SciPy \cite{arpack}.


\section{Calibration}


\subsection{Energy convergence}


The convergence of the eigenvalues with the grid size was initially studied for a zero potential. These results were later confirmed for a known potential. The calculations were performed on DIPC's Hyperion cluster \cite{hyperion}, with 16 cores and a varying RAM size up to 1Tb for the largest grid sizes.\\

A reasonable convergence of 3 decimals was observed for the energies with quantum numbers n=1 and n=2 with grids of 5,000 and 20,000 respectively. For a convergence of 4 decimals, for n=1 and n=2 the grids grow up to 50,000 and 200,000 respectively. The results from the convergence study are summarized in the following table. A comparison for a known potential \cite{titov2023} between a converged and an oversized grid is presented. Notice that the eigenvalues in the outputs are labeled without degeneration for computational purposes.

\begin{table}[H]
    \centering
    \begin{tabular}{ccclrr}
        \textbf{Quantum n. (n)} & \textbf{Eigenvalue} & \textbf{Decimal precision} & \textbf{Gridsize} & \textbf{Runtime} \\
        0  & 0     & -   & -             & -         \\
        1  & 1, 2  & 3   & 5,000         & 0.6s      \\
        1  & 1, 2  & 4   & \bf{50,000}   & 3min      \\
        2  & 3, 4  & 3   & 20,000        & 13s       \\
        2  & 3, 4  & 4   & \bf{200,000}  & 1h 45min  \\
        3  & 5, 6  & 3   & 50,000        & 3min      \\
        4  & 7, 8  & 3   & 90,000        & 9min 30s  \\
        5  & 9     & 3   & 150,000       & 40min
    \end{tabular}
    \end{table}

\NEWPICC{titov2023_2.png}{1}
\NEWPICC{titov2023_op_2.png}{1}

\NEWPICC{convergence_E2_00004.png}{1}


\subsection{Interpolation of custom potentials}


QRotor reads input potentials as \texttt{*.dat} files, with angle and energy information.
This potential data can be previously calculated with electronic structure methods. The potential can then be interpolated to a bigger grid size to converge the energies. To do so, it takes advantage of SciPy's Cubic Splines, where the interpolated curve is made of points with matching first and second derivatives \cite{cubicsplines}.\\

To validate the interpolation method, a test potential with 360 points was capped to 36 points. An interpolation to its previous size proved to match the original potential. Further interpolation to a grid of size 5000 showed the expected convergence between the original and the capped potential up to the 3rd decimal place. The results are shown in the following figures.

\newpage

The original potential:
\NEWPICC{interpolate_potential_360.png}{1}
And the capped potential:
\NEWPICC{interpolate_potential_36.png}{1}

\newpage

The original potential, interpolated to a grid of size 5000 to ensure a convergence up to the 3rd decimal position:
\NEWPICC{interpolate_potential_5000.png}{1}
As expected, the interpolated capped potential converges up to the 3rd decimal position:
\NEWPICC{interpolate_potential_5000-from-36.png}{1}


\section{Preliminary results}


\subsection{Isotopic effects}


It is expected from eq. \eqref{b} to observe a lower rotational energy as a result of an increase in the atomic masses. The isotopic effects on the energies of the hindered methyl rotor were studied by replacing the hydrogen mass with that of deuterium. A test potential was interpolated to a grid of size 200000 to ensure an optimal convergence. As expected, the results show a shift to lower energies for the deuterated counterpart.

\NEWPICC{interpolate_potential_HD.png}{1}


\subsection{Custom potential: \texttt{potential\_felix.dat}}


A custom potential provided by Félix was studied. The potential is given in eV, with a moment of inertia of 301.54 amu \AA$^2$. The rotational energy B was calculated as
$$
B = \frac{1}{2 I} = 1.658 \times 10^{-3} \text{eV}
$$

The potential was interpolated to a grid of size 200000 to ensure an optimal convergence. The results are shown in the following figure.

\NEWPICC{potential_felix.png}{1}


\section{Comment on why to solve the eigenvalues for every new B value}


Note that from the Hamiltonian eq. \eqref{hamiltonian},

\begin{equation*}
    H = -B \frac{d^2}{d\varphi^2} + V(\varphi)
\end{equation*}

It can be deduced that B can theoretically be extracted by expressing the potential as a reduced variable, so that

\[
    \frac{H}{B} =
    \left[ {\begin{array}{cccccc}
     +2 +\frac{V_1}{B}  &  -1                &  0     & \cdots &  0                     &  -1                \\
     -1                 & +2 +\frac{V_2}{B}  &  -1    &        &  0                     &  0                 \\
     0                  &  -1                & \ddots &        &                        &  \vdots            \\
     \vdots             &                    &        & \ddots &  -1                    &  0                 \\
     0                  &  0                 &        &  -1    & +2 +\frac{V_{N-1}}{B}  &  -1                \\
     -1                 &  0                 & \cdots &  0     &  -1                    & +2 +\frac{V_N}{B}  \\
    \end{array} } \right]
\]

But this matrix can not be solved numerically until \textit{B} is known.
\textit{B} cannot simply be set to 1 and then multiply the Hamiltonian eigenvalues by the desired \textit{B}.\\

Knowing the convergence criteria, it does not suppose any problem to just repeat the calculation for a new \textit{B} value.


\addcontentsline{toc}{section}{References}
\bibliographystyle{IEEEtran} % Estilo
\bibliography{QRotor_refs} % documento .bib


\end{document}

